\section*{Proforma}
\begin{tabular}{ll}
    Candidate Number & 2405E \\
    Title & Embedding Pointful Array Programming in Python \\ 
    Examination & Computer Science Tripos -- Part II, 2024 \\
    Word Count & 11993\footnote{Computed with \texttt{texcount}.} \\
    Code Line Count & 7071\footnote{Computed with \texttt{cloc}.} \\
    Project Originator & The candidate \\
    Project Supervisor & Professor Alan Mycroft \\
\end{tabular}

\subsection*{Original aims of the project}
Machine learning and scientific computing ecosystems are dominated by implementations of the \textit{array programming model} in Python, such as NumPy.
Writing and maintaining programs in the model is known to be difficult.
However, significant engineering effort has been spent on making it efficient.

This project set out to investigate \textit{pointful array programming} as an alternative by designing and implementing a pointful array language embedded in Python. 
Furthermore, the project would explore methods to execute it with Python's established array libraries. 
Thus, it would reconcile the expressiveness of pointful array programming, and performance of the array programming model.

\subsection*{Work completed}

The project was a complete success. All success criteria were met with greatly extended scope, with the designed embedded language -- Ein -- implementing various array programming features previously unavailable in Python. An efficient compilation scheme from pointful array programming to the array programming model was developed, relying on a new formal connection between the two styles. 

My project won first prize among undergraduates in the ACM Student Research Competition at the 51st POPL conference. Furthermore, a research paper coauthored with my supervisor was accepted for publication in the 10th ACM ARRAY proceedings.

\subsection*{Special difficulties}
None.
