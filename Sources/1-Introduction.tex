\chapter{Introduction}

In his Turing Award Lecture, \textit{Notation as a Tool of Thought}, Iverson argues for the importance of programming language design through the lens of mathematical notation  \cite{iverson2007notation}. The \textit{array programming model} he introduced in APL is still in use today thanks to its performance and conciseness. The ideas presented by it were revolutionary, particularly the brevity of the language and expressiveness despite its relatively simple grammar. It contrasts in many ways with numerical languages of its time, such as FORTRAN -- as Smillie describes in \textit{Discovering Array Languages} \cite{smillie2000lecture}. 

APL's ideas certainly stood the test of time in its various implementations and the languages it influenced, such as J or MatLab. But this begs the question -- is the model still appropriate for use in modern applications such as deep learning, or is there need for a new notation?

\section{Motivation}

When I found myself in a research team working on generalisation of deep learning models, I was shocked to see novel model architectures attract Python implementations as troublesome as \href{https://github.com/google-deepmind/clrs/blob/8697f51663bd77548f4b3108816c84d163883361/clrs/_src/processors.py#L140}{this one}, which I paraphrase below in the style of \textit{NumPy:}
% FIXME: Currently, the snippet defines logits and values, but does not use the former. Also fix in Ein.
\begin{center} \begin{minipage}{0.65\textwidth}
\begin{minted}{python}
logits = (
    transpose(att_1, (0, 2, 1, 3)) +  # + [B, H, N, 1]
    transpose(att_2, (0, 2, 3, 1)) +  # + [B, H, 1, N]
    transpose(att_e, (0, 3, 1, 2)) +  # + [B, H, N, N]
    expand_dims(att_g, axis=[-1, -2]) # + [B, H, 1, 1]
)                                     # = [B, H, N, N]
values = transpose(values, (0, 2, 1, 3))  # [B, H, N, F]
ret = matmul(c, values)  # [B, H, N, F]
return transpose(ret, (0, 2, 1, 3))  # [B, N, H, F]
\end{minted}
\end{minipage} \end{center}
Problems found in this snippet are not uncommon in the domain. The code was difficult to write and is not much easier to read, there are plenty of cryptic parameters that are \textit{probably correct}, and it was necessary to add comments. These issues can be attributed to a failure of the underlying paradigm -- the array programming model. \textcite{paszke2021getting} argue why the model might not be an apt abstraction for modern workflows. Though there are certainly benefits, such as the abundant parallelism present and usually good behaviour under automatic differentiation, there are also shortcomings. In short, the notation is often too explicit while also being too unconstrained, leading to code unreadable by both humans and machines. 

Matters get worse when one realises the deep learning ecosystem is extremely centralised -- nearly all research and a significant part of engineering takes place in Python in a select few libraries. Examples include PyTorch, TensorFlow, Jax, and they turn out to all derive from the aforementioned NumPy, which embraces the array programming model. 
% Even attempts at interoperability like the ONNX standard end up using the same paradigm, preserving the status quo \cite{jin2020compiling}. 
This makes implementing high-performance compilers unexpectedly hard, leading to the introduction of different intermediate representations, posing challenges in translation to different paradigms \cite{feng2023tensorir}. The deep learning ecosystem is also constantly evolving, with new operations and hardware targets constantly developed, making for a colossal engineering effort in any single project. This slows down development of any new practical approaches.

\section{Prior art}

Typical examples of languages used for numeric programming include C/C++, FORTRAN, or Julia. They tend to have influences from APL (in design \cite{bernecky1991fortran} or packages \cite{eigenweb}), but are primarily performance-focused imperative languages -- which do grant higher degree of control at the expense of the need for more challenging compilation schemes to produce efficient programs \cite{grosser2012polly}. The situation is similar when programming hardware accelerators such as GPUs with the C family languages. Importantly, all of these are seldom applied in deep learning directly, and rather they are called from Python via its foreign function interface. One usually only resorts to them when the existing frameworks fail.
% C/C++ are exceptions to the rule, due to their interoperability with Python and impact on GPGPU programming -- hence much of the numerical heavy-lifting is done with them. 

Issues of the array programming model have not gone unnoticed in the Python community. Many of the problems revolve around higher-dimensional arrays, which have become commonplace in the domain. And similar to Iverson's approach, mathematical notation was sought after as a inspiration for an alternative -- in this case it was \textit{Einstein summation}. Initially limited approaches in Python -- such as \textit{Tensor Comprehensions} \cite{vasilache2018tensor} and \texttt{einops} \cite{rogozhnikov2021einops} -- can be seen as the emergence of \textbf{pointful array programming}. This paradigm -- best exemplified by the Dex language \cite{paszke2021getting} -- aims to fix many problems of the established model, taking centre stage in this project. 

\section{Aims}

The main aim of the project is to design and implement a domain-specific language that pragmatically addresses problems of the array programming model, particularly in deep learning. This meant that:
\begin{itemize}
    \item The designed language needs to show improvement of notation on a selection of problems. It should be general enough to avoid awkward code switching and build on a formal foundation. There is a tool for every problem, and so we improve on the array programming model where it falls short. 
    \item Since Python is such an important aspect of the target domain, the finished product needs to be easy-to-use in it. Attention should be paid to practicality and a good programmer experience.
    \item Due to the significant effort needed to produce efficient array code targeting all kinds of hardware, the compiler should capitalise on existing efforts in established projects to achieve this universality.
\end{itemize}
Based on these criteria, I chose to shallow-embed a new pointful array language in Python, and to compile it to calls in the NumPy library in a fashion that is easy to generalise to similar targets. 

\paragraph{Foreshadowing} The motivating example translates to the following code in my new language:
\begin{center}
\begin{cminted}{python}
logits = array(
    lambda b, h, u, v: s[b, u, h] + t[b, v, h] + e[b, u, v, h] + g[b, h]
)
return array(
    lambda b, u, h, f: reduce_sum(
        lambda v: c[b, h, u, v] * values[b, v, h, f]
    )
)
\end{cminted}
\end{center}
The code became much closer to an index-oriented mathematical notation. All of the cryptic transpositions are gone, and the nature of the matrix multiplication became evident. Most importantly of all -- this alternative approach executes just as fast as the original, while being much more readable.
