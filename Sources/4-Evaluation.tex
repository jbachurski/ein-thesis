\chapter{Evaluation}

I conduct the evaluation of the project on the basis on the success criteria chosen in the proposal.
These were \textbf{expressiveness}, \textbf{correctness} and \textbf{performance}, which are all reasonable to expect from a new language and compiler. 
In addition, we consider Ein's \textbf{usability} as an alternative to writing array programs in NumPy and frameworks deriving from it.
Lastly, since I introduce novel methods in this project, we discuss the \textbf{research contribution} based on external reviews conducted throughout the course of the project.

\section{Expressiveness}

We evaluate the expressiveness of Ein from two perspectives: \begin{itemize}
    \item For a \textit{programmer}, what useful patterns are easy to express in Ein?
    \item \textit{Computationally}, what can or cannot be expressed in Ein's formal program representation, Phi?
\end{itemize}

\subsection{Programming in Ein}

\paragraph{Axis manipulation}

\paragraph{Indexing}

\paragraph{Batching}

\paragraph{Record types}

\paragraph{Functions}

\subsection{Computation in Phi}

\paragraph{Primitive recursion}

\paragraph{Limited control flow}

\paragraph{Lack of in-place updates}

\section{Correctness}

\subsection{Tests}

\section{Performance}

\subsection{Benchmarks}

\section{Usability}

\subsection{Comparison}

\subsection{Notation}

\section{Research contribution}

\subsection{Related work}

\subsection{Student Research Competition}

The ACM Student Research Competition (SRC) is an opportunity for undergraduate and graduate students to present their original research and have it reviewed by judges. 
It consists of two stages -- the conference stage, with winners from each conference invited to the ACM Grand Finals. 
Prior to participation, I only had experience with a student research competition prior at the high school level.

I submitted a 3-page (plus 5-page appendix) extended abstract to the SRC at the 51st ACM SIGPLAN Symposium on Principles of Programming Languages (POPL 2024), in the undergraduate track.
It concerned the core of my project with the limited extensions already implemented at the time.
This work was done in late November, receiving positive reviews in December. 
I was invited to present my work at the conference in January. 
I had to prepare a poster for presentation, and slides for a brief oral presentation. 
Throughout the entire process I was supported by my supervisor, who reviewed the materials I prepared.
My work won \textbf{first prize} among undergraduates. 

As an undergraduate winner of the POPL SRC, I was invited to submit a 5-page extended abstract to the ACM SRC Grand Finals in late April. Results for this stage are due later in May.

\begin{center}
    \textcolor{blue}{TODO: Comment on submission in late April? Results likely not prior to dissertation submission. }
\end{center}

\subsection{ARRAY Workshop}

The ACM SIGPLAN International Workshop on Libraries, Languages and Compilers for Array Programming (ARRAY) concerns all aspects of array programming. 
This includes topics particularly relevant to this project: language design, embedded DSLs, paradigms, and compilation schemes.
The 10th ARRAY workshop will be colocated with the ACM SIGPLAN Conference on Programming Language Design and Implementation (PLDI) in late June 2024.

As part of the proceedings of the workshop in early April, together with my supervisor we authored a 12-page research project on the topic of this work. I wrote the main content, with my supervisor advising me on the paper structure and taking up an editorial role. 

\begin{center}
    \textcolor{blue}{TODO: Comment on results in late April.}
\end{center}


\section{Summary}
