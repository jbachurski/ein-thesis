\chapter{Preparation}

\section{Array programming}

\subsection{Array programming model in NumPy}

Much of today's deep learning and scientific computing workflows takes place in the \textit{array programming model}, as introduced in APL by \textcite{iverson1962programming}. The leading Python library for efficiently processing multidimensional arrays, NumPy, is no exception. 

\subsubsection{Broadcasting} \textbf{Universal functions} (ufuncs)

\subsubsection{Shape manipulation}

\subsubsection{Reductions}

\subsection{Pointful array programming}

\subsubsection{Einstein summation}

\subsubsection{Dex}

\subsection{Role of Python}

\subsection{Mapping to hardware}

\subsection{Other approaches}

\subsubsection{Named tensors}

\subsubsection{Functional -- Futhark}

\subsubsection{Macro-based}

\section{Domain-specific languages}

\subsection{Embeddings}

\subsubsection{Tracing}

\subsubsection{Operator overloading}

\subsection{Python DSLs}

\section{Functional programming concepts}

\subsection{Pointful and point-free programming}

\url{https://en.wikipedia.org/wiki/Tacit_programming}

\subsection{Applicative functor}

\subsection{Representable functor}

\subsection{Total programming}

\section{Compiler techniques}

\subsection{Redundancy elimination}

\subsubsection{Common subexpression elimination}

\subsubsection{Loop-invariant code motion}

\subsection{Equivalence classes}

\subsection{Abstract interpretation}

\subsection{Dependent types}

\section{Starting point}

\section{Requirements analysis}

\section{Software engineering}

\subsection{Methodology}

\subsection{Choice of tools}

\subsection{Review of test suites}

\subsection{Licensing}
